\chapter{Reflections and Conclusion}
\label{sec:reflection_conclusion}

\section{Related Work}
\label{sec:related_work}

This work exists in the field of formal verification of cryptographic protocols.
Notably our work has been heavily influenced by similar formalisations
\cite{cryptoeprint:2019:1185,DBLP:journals/corr/MetereD17,certicrypt_sigma,zkcrypt,Yao}

Sigma protocols has been done in a lesser extend in \easycrypt. Much of the same
work has been done in Isabelle/CryptHOL by \citeauthor{cryptoeprint:2019:1185}.

\citeauthor{certicrypt_sigma} formalised $\Sigma$-Protocols within CertiCrypt,
and proved the security of the $\Sigma^{\phi}$-Protocol, which proves knowledge
of the pre-image of a group homomorphism. ZKBoo protocol described in section
\ref{sec:zkboo} and the $\Sigma^{\phi}$-Protocol prove knowledge of the same
relation, but ZKBoo has reduced proof size? \todo{Examine differences}.

phi assumes the group homomorphism to be special?

\todo{Differences between the works}

Commitment schemes has been formalized in \easycrypt by
\citeauthor{DBLP:journals/corr/MetereD17}.

\todo{Differences between the works}

formalised general zero knowledge compilers have been explored, with some
notable work by \citeauthor{zkcrypt} and PINOCCIO.

\todo{Differences between the works}

\todo{SFE in EC, YAO}

\section{Discussion}
\label{sec:discussion}

\todo{How has EC been to work with}
\todo{What is the future for cryptographers using EC}
\todo{Possible code extraction?}
\todo{Schism between perfect and computation distinguishably}

\todo{Adversary based games are not always ideal}
\todo{Not possible to swap order of verify in commitment if procedure}

\section{Future work}
\label{sec:future_work}
In this thesis has created a workable formalisation, as show by the formalisation of
ZKBoo. Various improvement has then been made to the ZKBoo protocol to mainly
reduce to proof size but also to provide zero-knowledge in a post-quantum
context \cite{zkb++}.

With our formalisation we have intentionally focused on the ZKBoo protocol in
isolation but in real applications it would be part of a larger tool chain.
Mainly, ZKBoo requires a circuit with a definable execution order to be secure.
In our formalisation we have assumed the input to be a circuit and defined an
execution order but to complete the tool chain we would need a formalisation of
a procedure converting functions to circuits and a formal proof of the induced
execution order in section \ref{subsec:arith-representation} being semantic preserving.

Moreover we saw in section \ref{subsec:fiat-shamir} that there is a need for
formalising the rewinding lemma to reason about soundness of the Fiat-Shamir
transformation. Moreover, rewinding is a common technique for proving soundness
of zero-knowledge protocols. Formalising the rewinding lemma would then allows
us to reason about be general zero-knowledge protocols than the sub-class of
$\Sigma$-Protocol which we have explored in this thesis.

\todo{Prove connection between $\Sigma$ and pok or arg ZK}

\section{Conclusion}
\label{sec:conclusion}

\todo{conclude on the problem statement from the introduction}

%%% Local Variables:
%%% mode: latex
%%% TeX-master: "../main"
%%% End:
