\chapter{Reflections and Conclusion}
\label{sec:reflection_conclusion}

\section{Related Works}
\label{sec:related_works}
This work exists in the field of formal verification of cryptographic protocols.
Notably our work has been influenced by similar formalisations
\cite{cryptoeprint:2019:1185,DBLP:journals/corr/MetereD17,certicrypt_sigma,zkcrypt,Yao}

\cite{cryptoeprint:2019:1185} formalises both $\Sigma$-Protocols and
commitment schemes within Isabelle/CryptoHOL. Additionally, they formalise
the proof that commitment schemes can be build directly from $\Sigma$-Protocols.
Their formalisation of $\Sigma$-Protocols also include various concrete instantiations.
The main difference between the results obtained in their work compared to ours
has been the tool usage. Isabelle/CryptHOL is a tool similar to \easycrypt\ that
offers a higher-order logic for dealing with cryptographic game-based proofs.
There are two fundamental difference between the two tools.
First, Isabelle/CryptHOL programs are written in a functional style, where as
\easycrypt\ allows the user to write programs in an imperative style. This
ultimately leads to the same understanding of programs as distribution
transformers as discussed in chapter \ref{ch:EasyCrypt}.

Second, CryptHOL is a foundational framework built on top of Isabelle, which
means that all CryptHOL rules are derived from the underlying rules of Isabelle.
To trust CryptHOL, it is, therefore, enough to trust Isabelle, which is a much
larger project with more contributors. \easycrypt, however, is a strand-alone
tool and depends on axioms and SMT solvers to validate its deduction rules.
The benefit of this, however, is that by having a tighter integration with SMT
solvers we can achieve higher levels of automations. The downside, however, is
that the trusted computing base of \easycrypt is larger.

Other formalisations of $\Sigma$-Protocols also exists. \cite{certicrypt_sigma}
successfully formalised $\Sigma$-Protocols using CertiCrypt. Their work include a
formalisation of $\Sigma$-Protocols where the relation is the pre-image of a
homomorphism with certain restrictions or a claw-free permutation. This has
allowed them to define and prove the security for a whole class of
$\Sigma$-Protocols.
This result is similar to the one we achieved with our
formalisation of ZKBoo. ZKBoo, however, works on a less restrictive class of
$\Sigma$-Protocols than the one defined in this paper.

Moreover, commitment schemes have been formalised in \easycrypt\ by
\cite{DBLP:journals/corr/MetereD17}. Their work differs from ours by offering
fewer definitions of security, which we described in section
\ref{sec:commitment:alt-sec} and show the need of in the special soundness and
SHVZK proofs in section \ref{subsec:zkboo:sec}

Notable work also exists for formalising generalised zero-knowledge compilers.
\cite{zkcrypt} developed a verified zero-knowledge compiler in CertiCrypt
which uses the generalised Schnorr protocol to produce zero-knowledge proofs of
any relation defined by the pre-image of a group homomorphism, just like ZKBoo.
The generalised Schnorr protocol, however, is a fundamentally different protocol
than ZKBoo, in the sense that it does not use MPC or commitment scheme, which
also contribute to vastly different communication costs.

Last, numerous formalisation of Multi-Party Computations exists.
\cite{Yao} formalised secure function evaluation with Yao's protocol in
\easycrypt.
Their work also included a formalisation of circuits.
\cite{DBLP:journals/corr/abs-1805-12482} formalised a two party MPC protocol
based on secret sharing against a semi-honest adversary. Their work also manage
to formally prove the privacy property with a simulator, akin to our work on the
(2,3)-Decomposition in section \ref{sec:decomposition}.
\cite{DBLP:journals/corr/abs-1806-07197} then formalised security of
secret-sharing based MPC, but in the presence of an active adversary in \easycrypt.

\section{Discussion}
\label{sec:discussion}
Throughout the work of this thesis we have used the \easycrypt\ proof assistant
to formally verify the proofs presented herein.
The work has been an iterative process between formulating a lemma and then
trying to prove it within \easycrypt. This process was then repeated until the
lemma would be formally proven. Consequently, \easycrypt\ has been a
instrumental part of the work formulated in this thesis. Moreover, the powerful
automation tools offered by \easycrypt\ has allowed us to discharge trivial
proofs, thus enabling us to spend more time working on the complex lemmas seen
within this thesis.

Overall, we feel that \easycrypt\ captures the models used by cryptographers
quite well; our formalisation of ZKBoo has a structure akin to the one presented
in the original paper by \cite{zkboo}. This has enabled us to spend less time
formulating the protocol and more time on formalising important security criteria.
This has been capacitated, in part, by \easycrypt\ \textit{pWhile} language for
implementing procedures. This language follow a structure closely resembling the
pseudo-code seen in cryptographic papers.

Ultimately, the tool offers the possibility of writing programs both in a
functional style and in an imperative style. It is, however, only the programs
written in the imperative style that are allowed to make random choices.

Our main problems with using this tool has been the schism between perfect
indistinguishability proven in pRHL and indistinguishability based on an
adversarial game and the tool's steep learning curve.

In particular \easycrypt\ offers its rPHL for proving procedures to be perfectly
indistinguishable. If, however, computational indistinguishability is needed then
the rPHL logic cannot directly be used, and we instead have to deal with
adversaries comparing the procedures.

This problem is part of a more general problem where \easycrypt\ in essence
offers two techniques for dealing with cryptographic proofs. The first is the
traditional adversaries game-hopping technique where we reason about an
adversary being able to break the security of the protocol. These adversaries can
then be used to construct new adversaries that can break the security of other protocols.
The other techniques is showing indistinguishability of the output distributions
of the programs with \easycrypt's relational logic.
Both of these techniques are perfectly valid for proving security of
cryptographic protocols.
However, at the time of writing this thesis it is not
clear to us how to formally prove the relation between the two techniques.
The gap between the two techniques became apparent in chapter \ref{ch:formal_zkboo} where the
adversarial-based security definitions of our commitment schemes did not conform
to the goals needed to prove security of our $\Sigma$-Protocol formalisation.

The steep leaning curve is primarily caused by the lack of documentation of new
tactics. At the time of writing this thesis the last update to the \easycrypt\
reference manual \cite{ec_refman} was in 2018. Moreover, the deduction rules by
the different logics that \easycrypt\ provides are not documented anywhere, but
instead have to be found in the papers describing \texttt{CertiCrypt} which is
the Coq-based proof assistant antecedent to \easycrypt.
Overall, the work of this thesis proved that it is possible to formalise research papers in a
reasonable time frame with no prior knowledge of \easycrypt.

Other difficulties with \easycrypt\ have been described throughout this thesis.
In section \ref{sec:commitment:sec}, we outlines the problem of expressing
statistical security in \easycrypt. Then, in section \ref{sec:sigma_comp}, we
outlined the workaround needed to work with \easycrypt\ theories quantifying
over other theories. In section \ref{subsec:fiat-shamir} we outlined the problem
of rewinding in \easycrypt, which limits the constructions we can directly prove.

\section{Future work}
\label{sec:future_work}
In this thesis we have formalised $\Sigma$-Protocols and
commitment schemes that is applicable to larger cryptographic protocols, as shown
by our formalisation of ZKBoo. However, various improvement has since been made to the
ZKBoo protocol. Notably, the ZB++ protocol, which offers a reduction in the
size of messages/responses sent to the verifier from the prover. Moreover, it also provides zero-knowledge
in a post-quantum context \cite{zkb++}. An interesting next step could,
therefore, be to use our existing formalisation of ZKBoo to formally verify the
improvements made by ZB++.

With our formalisation we have intentionally focused on the ZKBoo protocol in
isolation, but in real applications it would be part of a larger tool chain.
Mainly ZKBoo requires a circuit with a definable execution order to be secure.
In our formalisation, we have assumed the function to represented as a circuit
and then defined an execution order. However, to complete the tool chain we would need a formalisation of
a procedure converting functions to circuits and a formal proof of the induced
execution order in section \ref{subsec:arith-representation} being semantics preserving.

Moreover, we saw in section \ref{subsec:fiat-shamir} that there is a need for
formalising the rewinding lemma to reason about soundness of the Fiat-Shamir
transformation. Moreover, rewinding is a common technique for proving soundness
of zero-knowledge protocols. Formalising the rewinding lemma would allow
us to reason about more general zero-knowledge protocols than the sub-class of
$\Sigma$-Protocol which we have explored in this thesis.

\section{Conclusion}
\label{sec:conclusion}
In this thesis, we have successfully managed to develop a rich formalisation of
$\Sigma$-Protocols and commitment schemes, while reproducing some of the key
results of formalisation done in other proof assistants.
Moreover, we have formalised a particular class of Multi-Party Computation
protocols on arithmetic circuits called the (2,3)-Decomposition.
This Multi-Party computational formalisation, in turn, has enabled us to formally
verify a generalised $\Sigma$-Protocol called ZKBoo.
In doing so, we showed how essential details for achieving security are
often glossed over in cryptographic literature.
Particularly we saw in section \ref{subsec:zkboo:sec} how important small
assumptions are for the security of implementations of cryptographic protocols.
If one procedure was allowed to observe the state of another running on the
system, then all proofs in the aforementioned section would not hold. These
assumptions are often left out when discussing cryptographic protocol design
but are essential when reasoning about the security of the protocols when
implemented in a programming language.

Additionally, we remarked on the usability of the \easycrypt\ proof
assistant and have shown that it is possible to formalise a research paper with
no prior knowledge of the tool.

%%% Local Variables:
%%% mode: latex
%%% TeX-master: "../main"
%%% End:
