\chapter{Reflections and Conclusion}
\label{sec:reflection_conclusion}

\section{Related Work}
\label{sec:related_work}

Sigma protocols has been done in a lesser extend in \easycrypt. Much of the same
work has been done in Isabelle/CryptHOL by \citeauthor{cryptoeprint:2019:1185}.

\citeauthor{certicrypt_sigma} formalised $\Sigma$-Protocols within CertiCrypt,
and proved the security of the $\Sigma^{\phi}$-Protocol, which proves knowledge
of the pre-image of a group homomorphism. ZKBoo protocol described in section
\ref{sec:zkboo} and the $\Sigma^{\phi}$-Protocol prove knowledge of the same
relation, but ZKBoo has reduced proof size? \todo{Examine differences}.

phi assumes the group homomorphism to be special?

\todo{Differences between the works}

Commitment schemes has been formalized in \easycrypt by
\citeauthor{DBLP:journals/corr/MetereD17}.

\todo{Differences between the works}

formalised general zero knowledge compilers have been explored, with some
notable work by \citeauthor{zkcrypt} and PINOCCIO.

\todo{Differences between the works}

\section{Discussion}
\label{sec:discussion}

\todo{How has EC been to work with}
\todo{What is the future for cryptographers using EC}
\todo{Possible code extraction?}
\todo{Schism between perfect and computation distinguishably}

\section{Future work}
\label{sec:future_work}
Thesis has created a workable formalisation, as show by the formalisation of ZKBoo. ZKBoo can be improved to make it more applicable in real applications \todo{Cite ZKBoo+ paper}

\begin{itemize}
  \item Explore rewinding in EC
  \item Other Zero-knowledge protocols
  \item Expand the toolchain
    \begin{itemize}
      \item Prove list representation equiv to graph
      \item Conversion from function to circuit (Metaprogramming?) ASTs not available? I/O not possible
      \item Proposed standard zk interfaces?
    \end{itemize}
\end{itemize}


\section{Conclusion}
\label{sec:conclusion}

\todo{conclude on the problem statement from the introduction}

%%% Local Variables:
%%% mode: latex
%%% TeX-master: "../main"
%%% End:
