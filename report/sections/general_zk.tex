\chapter{Towards generalised zero-knowledge compilation}
\label{ch:general_zk}
We have previously seen a concrete instantiation of a $\Sigma$-protocol with the
relation being the discrete logarithm problem, namely Schnorr's protocol. We
have also seen how it was possible to prove Schnorr's protocol to be secure in a
formal setting though \easycrypt.

There exists an infinite set of possible relations, for which we could want to
provide zero-knowledge proofs of. It is therefore infeasible to design a
protocol for each relation and proving its security.

An example of this is the GMW-compiler \dots, or the example system from \cite{zkcrypt}.

We therefore need a more generalised approach, that is able to generate
zero-knowledge proof for an entire family of relations rather than a specific relation.
One such family of relations is the pre-image under group homomorphisms \dots

Another important factor of zero-knowledge compilers is the reduce the proof
size. ZK proofs are notoriously expensive to run.

\section{ZKBOO}
\label{sec:zkboo}
The ZKBoo protocol, which was invented by \citet{zkboo}, is a zero-knowledge compiler for relations, which can be
expressed as the pre-image of a group homomorphism, i.e.

\[
  R \text{ h w} = \phi(w) = h
\]

This technique assumes that the function $\phi$ can be expressed in a way, that
allows for a (2,3)-Decomposition.

\todo{A (2,3) decomposition is...}

For the scope of this thesis we simply this assumption, and only look at
functions, which can be represented as a arithmetic circuit in some finite
integer ring. If the function is represented as an integer arithmetic circuit,
then a general technique exists for perform the (2,3)-Decomposition

\todo{Describe technique for 2,3 decomp of arith.}

\todo{Proof size of ZKboo}


%%% Local Variables:
%%% mode: latex
%%% TeX-master: "../main"
%%% End:
