\chapter{Introduction}
\label{ch:intro}
% \begin{draft}
%   Outline
%   \begin{itemize}
%     \item Why is zk important?
%     \item Why is formal verification important?
%     \item What has been done already?
%     \item How can it be improved upon?
%     \item What is the work of this thesis?
%     \item Why is it important
%     \item Formalisation of circuits?
%     \item MPC in the head
%     \item Use MPC to get zk?
%   \end{itemize}
% \end{draft}
Since their introduction zero-knowledge arguments/proofs has become
an important building block for complex cryptographic protocols.
Zero-knowledge have, for example, been used to enable anonymous board-room
voting \cite{boardroom-voting} and to ensure privacy in blockchain applications \cite{concordium}.
Moreover, zero-knowledge also has important theoretical applications, where it
is used by the GMW-compiler to turn any passively secure protocol into an
actively secure protocol.

Due to zero-knowledge being applied critical applications like
blockchains which could have real-life financial consequences, if the security is
broken, it is paramount that we can offer the highest levels of assurance in the
protocols performing to specification.
However, many proofs of security are considered unverifiable due their informal
reasoning and implicit arguments. This was also remarked by Bellare and Rogaway \cite{game-playing}:

\begin{quote}
In our opinion, many proofs in cryptography have become essentially unverifiable. Our field may be approaching a crisis of rigor.
\end{quote}

To harness this complexity Bellare and Rogaway suggests the code-based game-playing approach
for doing security proofs of cryptographic protocols.
However, game-playing security is not the end-all of security, and the approach
still allows for unrigorous proofs. Many cryptographers proofs still lends
themselves to informal arguments and human errors when proving security.

However, with the code-based approach, we can apply research from
formal verification to help us attain an even higher level of trust in our
proofs by making them machine verifiable.
This is approach has been made possible if the recent advance if proof assistant
specialising in formal verification of probabilistic programs \cite{SOK:CAC}.
It is important to remark, however, that formally verified does not imply
absolute security. When formally proving cryptographic protocols, we do so in
cryptographic models. These models help us reduce the complexity of our proves
by limiting external influences of the protocol or giving access to idealised
functionality. If the assumptions made in the model used are unsound, then the
protocol would still be unsecure if formally proven. However, By applying formal
reasoning to cryptographic proofs we can expel all informal reasoning and
implicit knowledge, thus attaining much more precise definitions of security.

An additional benefit of using tool supporting game-based security like \easycrypt\ is that
it becomes possible to formally verify protocols in a representation very close
to the one in cryptographic literature. This gives a more direct connection between
the formal proofs and how the protocols are used in practice.
These proof assistants also open the possibility of extracting the verified
protocol into an efficient language, which can be run on most computers.
One example of this is \easycrypt, where a low-level language called Jasmin has
been successfully embedded within \cite{easycrypt-jasmin}.
Cryptographic protocols written in a low-level machine language can
then, through EasyCrypt, be formally proven secure and extracted to assembly code.

This allows researchers to take a protocol description and then transfer their
definitions to a tool like \easycrypt, which can help expose informal reasoning
in their proofs.


In an attempt to bring higher assurance to the field of cryptography formal
verification of cryptographic primitives have commenced
\cite{SOK:CAC,ec_intro,cryptoeprint:2019:1185,certicrypt_sigma}.
Moreover, MPC...

The goal of this thesis is to further this formalisation effort by first improving the
existing formalisation of $\Sigma$-Protocols in \easycrypt.
$\Sigma$-Protocols has the desirable attribute of having code-based
game-playing definitions of security, which allows us to use \easycrypt, a
proof assistant for formal verification of cryptographic protocols, to formulate
and prove their security. Moreover, $\Sigma$-Protocols can be turned into
zero-knowledge proofs with limited overhead \cite{zkboo}.
To achieve this, we reproduce the results of \citet{cryptoeprint:2019:1185} to
develop a rich formalisation of $\Sigma$-Protocols in \easycrypt, which can be
used to drive the formalisation effort of cryptographic protocols further.

Additionally we reproduce the formalisation of commitment schemes by
\citet{DBLP:journals/corr/MetereD17} and improve upon the work by offering
alternative, yet equivalent, security definitions.

We then use our formalisation of $\Sigma$-Protocols to a formalise ZKBoo by
\citet{zkboo}, which uses multi-part computations and commitment schemes to make
a generalised $\Sigma$-Protocol. With this work we aim to show how complex
cryptographic protocol can be formally verified by using formalisations of
cryptographic primitives. Moreover, we show how formal verification can help expose
the implicit and informal assumptions often made in cryptographic papers.

To formalise ZKBoo we also have formalise arithmetic circuits, which can help guide
future formalisations. Moreover, formalisation of a special case of multi-party
computation, where all parties are simulated locally, called the
(2,3)-Decomposition is needed.


\paragraph{Outline}
In \autoref{ch:EasyCrypt}. we introduce the code-based game playing approach to
cryptographic security. Moreover, we introduce the \easycrypt\ proof assistant,
and how it can be used to formally prove code-based protocols.
\autoref{ch:background} introduce key concept and definitions.

In \autoref{ch:formal_commitment} we reproduce the results of
\citet{DBLP:journals/corr/MetereD17} to formalise commitment schemes and a
concrete instantiation in the form of Pedersen's commitment scheme.
Additionally, we formulate alternative security definitions, which can simply
using this formalisation in more complex protocols.

In \autoref{ch:formal_sigma} we explore the results by
\citet{cryptoeprint:2019:1185}, which were formalised in the Isabelle/CryptHOL
proof assistant. There results are then reproduced in \easycrypt.

\autoref{ch:general_zk} introduces generalised zero-knowledge protocols and
ZKBoo and an informal overview of the security proofs of ZKBoo given by \citet{zkboo}.

Next, in \autoref{ch:formal_zkboo} we formalise arithmetic circuit and the
(2,3)-Decomposition of arithmetic circuits. We then conclude the chapter by
formally proving the security of ZKBoo.

Last, we summarise our findings and their relation to other works within the
area of formal verification.

\paragraph{Implementation}
The \easycrypt\ files of the work explored in this thesis
are publicly available on Github\footnote{https://github.com/Nsidorenco/thesis}

%%% Local Variables:
%%% mode: latex
%%% TeX-master: "../main"
%%% End:
