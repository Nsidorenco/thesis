\chapter{Background}
\label{ch:background}


\section{Sigma Protocols}
\label{sec:sigma_protocols}
Originally introduced by Cramer, $\Sigma$-protocols are protocols on a
three-move-form, based on a, computationally hard, relation $R$, such that $(h, w) \in R$
if $h$ is an instance of a computationally hard problem, and $w$ is
the solution to $h$. $\Sigma$-protocols then allows a prover, P, who knows the
solution $w$, to convince a verify, V, of the existence of $w$, without
explicitly showing $w$ to him.

The following section aims to introduce the definition of $\Sigma$-protocols,
along with its notions of security. The following section is based on the
presentation of $\Sigma$-protocols by \citet{on-sigma-protocols}.

\subsection{Definition}
\label{sec:sigma_definition}
\begin{itemize}
  \item 3-move form
  \item Completeness
    \begin{itemize}
      \item Protocol should always succeed with correct output, if both parties are honest.
    \end{itemize}
  \item Special soundness
    \begin{itemize}
      \item Given two transcripts, with different challenges it is possible to
        compute an accepting witness for the statement in the relation.
    \end{itemize}
  \item Special Honest Verifier Zero-Knowledge.
    \begin{itemize}
      \item Exists an polynomial-time simulator M.
      \item Given statement, x, and challenge, e, output an accepting
        conversation (a, e, z).
      \item Conversation should have the same distribution as an conversation
        between honest parties.
    \end{itemize}
\end{itemize}

\section{Commitment Schemes}
\label{sec:commitment}

\subsection{Definition}
\label{sec:commitment:definition}



%%% Local Variables:
%%% mode: latex
%%% TeX-master: "../main"
%%% End:
