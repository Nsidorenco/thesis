\chapter{Background}
\label{ch:background}
In this chapter, we introduce the different fields of cryptography used
throughout this thesis.
Most of these fields have intricate security definitions which depend on the
context in which they are applied.
We, therefore, do not try to give a complete introduction to the fields.
Instead, we limit our introduction to the specific definitions relevant to the work of this thesis.

Notably, the section about multi-part computations (MPC) is intentionally left
brief.
The security definitions of an MPC protocol depends entirely on how much power
the different participants of the protocol has. For the work of this thesis we
only use the definitions of security where it assumed that all participants
follow the protocol. This is commonly refereed to as semi-honest security.

Additionally, in cryptography, it is common practice to quantify every security
definition over a ``security parameter''. This parameter exists to define how
hard specific computation problems should be. Consider, for example, the discrete
logarithm problem in a cyclic group. Naturally, the larger the group is,
the harder the problem is to solve since there is a larger set of potential solutions. The
security parameter would, in this case, be the size of the group.
The existence of the security parameter usually offers a trade-off between
security and efficiency. Considering the previous example for the discrete
logarithm; any protocol operating on a group will have arithmetic operations
with running time proportional to the size of the group. Therefore, the smaller
the group, the faster the protocol is to run in practice.

The security parameter, however, is usually left implicit, which we have also
done in the definitions below.

\paragraph{Notations}
While this thesis has been performed entirely in \easycrypt, the code in this
thesis will be given in a more pseudo-code style. For
the most part, we will avoid \easycrypt\ specific notation when writing procedures
and solely focus on the tools \easycrypt\ provides for reasoning about procedures.

Most notably we adopt the list indexing notation of $l[i]$ to mean the i'th
index of the list $l$. Formally, this notation is not sound since it does not
specify what will happen if the index does not exist. The is solved in \easycrypt\
by declaring a default element should the indexing fail. We omit
the default value from our code examples.
Moreover, we define $x::xs$ to mean that we prepend $x$ to the list $xs$,
$l_{1} ++ l_{2}$ means we concatenate two lists, and $\texttt{rcons} \; x \; xs$ means we
append x to the list $xs$.

Lastly, when referring to indistinguishability, we are referring the perfect
indistinguishability unless explicitly stated otherwise.

\section{Zero-knowledge}
\label{sec:background:zero-knowledge}
Zero-knowledge can be separated into two categories: \textit{arguments} and
\textit{proofs-of-knowledge}. We start by defining the former.

A Zero-knowledge argument is a protocol run between a probabilistic polynomial
time (PPT) prover P and a PPT verifier V. The prover and verifier then both
know a relation $R \subseteq \{0,1\}^{*} \times \{0,1\}^*$, which expresses a
computational problem. We refer to the first argument of the relation as $h$ and
to the second argument as $w$.
The goal of the protocol is then for P to convince V that he knows a pair
$(h,w)$ while only revealing $h$. At the end of the protocol, the verifier will
then output \textbf{accept} or \textbf{reject} based on whether P convinced
him or not.
We require that a verifier following the protocol always output
\textbf{accept} if P knew $w$ and followed the protocol.
This is known as \textit{correctness}.
Moreover, we require that a cheating adversary, who does not know $w$, can
only make the verifier output \textbf{accept} with some negligible probability $\epsilon$.

\textit{Proofs-of-knowledge} shares the same definitions
as above, but require that the verifier only output \textbf{accept} if the
prover indeed knew the pair $(h,w)$. Formally, the \textit{proof-of-knowledge}
variant is proven by assuming the prover has infinite computational power.


Common amongst both variants is that they require that verifier learns no
information, whatsoever, about w:

\begin{definition}[Zero-knowledge from \citet{on-sigma-protocols}]
  \label{def:zk}
  Any proof-of-knowledge or argument with parties (P, V) is said to be
  zero-knowledge if for every PPT verifier V$^{*}$ there exists a
  simulator $\text{Sim}_{V^{*}}$ running in expected polynomial time which outputs
  conversations indistinguishable from a real conversation between (P, V$^{*}$).
\end{definition}


\section{Sigma Protocols}
\label{sec:background:sigma_protocols}
The following section introduce $\Sigma$-protocols,
and their definition of security. The definitions given in the section is
inspired by presentation of $\Sigma$-protocols by \citet{on-sigma-protocols}.
$\Sigma$-protocols are two-party protocols on a three-move-form, with a
computationally hard, relation $R$, such that $(h, w) \in R$ if $h$ is an
instance of a computationally hard problem, and $w$ is the solution to $h$.
This relation can also be expressed as a function, such that
$(h,w) \in \text{R} \iff \text{R}_{f} \; h \; w = 1$.
A $\Sigma$-protocol allows a prover who knows $w$ to convince a
verify, without revealing $w$ to him.


\begin{figure}[ht]
  \centering
  \begin{tikzpicture}
      % \draw (-3,0) -- (-3,-3) (3,0) -- (3,-3);
      \node at (-3,.3) {Prover};
      \node at (-3,-.2) {$(h, w)$};
      \node at (3,.3) {Verifier};
      \node at (3,-.2) {$h$};
      \node at (-5,-1) {$Init(h, w) \rightarrow a$};
      \draw[->] (-3,-1) -- node[midway,above] {a} (3,-1);
      \draw[<-] (-3,-2) -- node[midway,above] {e} (3,-2);
      \node at (-5.75,-2) {$Response(h, w, a, e) \rightarrow z$};
      \draw[->] (-3,-3) -- node[midway,above] {z} (3,-3);
      \node at (5,-3) {$Verify(h,a,e,z)$};
  \end{tikzpicture}
  \caption{\label{fig:proto_sigma} $\Sigma$-Protocol}
\end{figure}

A general overview of a $\Sigma$-Protocol can be seen in figure
\ref{fig:proto_sigma}. Here we note that the protocol is on a three-move form
since only three messages, $(a,e,z)$, are sent between the prover and the verifier.

\paragraph{Security}
A $\Sigma$-Protocol is said to be secure if it satisfy the following definitions.

\begin{definition}[Completeness]
  Assuming both P and V are honest \ie\ following the protocol, then V will always \textbf{accept} at
  the end of the protocol.
\end{definition}

\begin{definition}[Special Soundness]
Given a $\Sigma$-Protocol $S$ for some relation $R$ with public input $h$
and two any accepting transcripts $(a,e,z)$ and $(a,e',z')$
where both transcripts have the same initial message, $a$ and $e \neq e'$.

Then $S$ satisfies $2$-special soundness given an efficient
algorithm, called the ``witness extractor'' exists, that
given the two accepting transcripts outputs a valid witness for the relation $R$.
\end{definition}

The special soundness property is important for ensuring that a cheating prover
cannot reliably convince the verifier.
Given special soundness property, the cheating prover probability of convincing
the verifier becomes negligible if the protocol is run multiple times.
Special soundness implies that there can
only exist one challenge, for any given message $a$, which can make the
protocol accept, without knowing the witness. Therefore, given a challenge space
with cardinality $c$ the probability of a cheating prover succeeding in
convincing the verifier is $\frac{1}{c}$. The protocol can then be run multiple
times to make his probability $\frac{1}{c}^{n}$, where n is the amount of runs.

The special soundness definition can also be generalised to $s$-Special
Soundness. This definition requires that the witness can be constructed, given
$s$ accepting conversations.

\begin{definition}[Special honest-verifier zero-knowledge]
  A $\Sigma$-Protocol $S$ is said to be SHVZK if there exists a polynomial-time
  simulator Sim which given instance $h$ and challenge $e$ as input produces a
  transcript $(a,e,z)$ indistinguishable from the transcript produced by $S$
\end{definition}

$\Sigma$-Protocols have the convenient property of being able to construct
zero-knowledge protocols from any secure $\Sigma$-Protocol in the random oracle
model with no additional computations. This effectively allows us to construct a
secure zero-knowledge protocol while only having to prove that the protocol is
zero-knowledge in the case of an honest verifier. This transformation from
$\Sigma$-Protocol to zero-knowledge protocol is known as the ``Fiat-Shamir
transformation''. More details about this transformation can be found in section
\ref{subsec:fiat-shamir}.
Moreover, it is possible to turn any $\Sigma$-Protocol into a zero-knowledge
argument with one additional round of communication between the Prover and
Verifier or a proof-of-knowledge with two extra rounds of communication without
assuming access to a random oracle \cite{on-sigma-protocols}.


\section{Commitment Schemes}
\label{sec:background:commitment}
Commitment schemes are another fundamental building block in cryptography, and
has a strong connection to $\Sigma$-Protocols where it is possible to construct
commitment schemes from $\Sigma$-Protocols \cite{cryptoeprint:2019:1185}.
A commitment scheme facilitates interaction between two parties, the committer
and the verifier.
The committer has a message, m, which he can use to generate a commitment and
send it to the verifier.
At a later point, C can then reveal the message
to V, who can then use the commitment to verify the message C revealed is indeed
the same message he used to generate the commitment.

\begin{definition}[Commitment Schemes]
A commitment scheme is a tuple of algorithms (Gen, Com, Ver), where:
\begin{itemize}
  \item $(ck, vk) \leftarrow Gen()$, provides key generation.
  \item $(c, d) \leftarrow Com(ck, m)$ generates a commitment $c$ of the message
    $m$ along with an opening key $d$ which can be revealed at a later time.
  \item $\{true, false\} \leftarrow Ver(vk, c, m, d)$ checked whether the
    commitment $c$ was generated from $m$ and opening key $d$.
\end{itemize}
\end{definition}

For commitment schemes to be secure; it is required to satisfy three
properties: \textbf{Correctness}, \textbf{Binding}, and \textbf{Hiding}.

\begin{definition}[correctness]
  A commitment scheme is said to be correct if
  the verification procedure  will always accept a commitment made by an honest party,
 \ie
$$
Pr[Ver(vk, c, m, d) = true | (c, d) = Com(ck, m) \land (ck, vk) \leftarrow Gen()] = 1.
$$
\end{definition}

\begin{definition}[binding]
The binding property states that a party committing to a message cannot
convince the verifier that he has committed to a message different from the
original message, \ie $(c, d) \leftarrow Com(ck, m)$, will
not be able to find an alternative opening key $d'$ and message $m'$ such that
$(c, d') \leftarrow Com(ck, m')$.

The scheme is said to have \textit{perfect binding} if it is impossible to change the
opening.
\textit{statistical binding} is achieved if there is a negligible probability of changing
the opening and \textit{computation binding} if producing a valid opening to a
different message is equivalent to a hard computation problem.
\end{definition}

\begin{definition}[hiding]
The Hiding property states that a party given a commitment $c$ will not be able
to guess the message $m$ on which the commitment was based.

The scheme is said to have \textit{perfect hiding} if it is impossible to
distinguish two commitment of different messages from each other,
\textit{statistical hiding} if there is a negligible probability of
distinguishing the commitments and \textit{computational hiding} if
distinguishing the commitments is equivalent to a hard computational problem.
\end{definition}


\section{Multi-Part Computation (MPC)}
\label{sec:background:mpc}
Consider the problem with $n$ parties, called $P_{1}, \dots, P_{n}$, with
corresponding input values $\textbf{x} = x_{1}, \dots, x_{n}$ where the parties
are allowed to communicate with each other over a secure channel freely.
The parties then want to compute a public function,
$f : (\text{input})^{n} \rightarrow \text{output}$, where each party contribute
their input to the function and after computing the function every party will
have the same output. However, none of the participant are allowed to learn
about any of the inputs to the function, barring their own.

To achieve this, the parties jointly run an MPC protocol $\Phi_{f}$. This protocol
is defined in the term of rounds.
In each round, each party $P_{i}$ computes
a deterministic function of all previously observed messages.
The party then either sends the computed value to another party or broadcasts it to everyone.
We define the collection of computed values and received values for party $i$ as $\text{view}_{i}$.

Once all rounds of the protocol have been completed, the output values $y$ can be
directly computed based on $\text{view}_{i}$.

% \todo{Yao's millionaire problem as primer?}

% \todo{semi-honest security}

\begin{definition}[correctness]
  \label{def:mpc:correctness}
  An MPC protocol $\Phi_{f}$ computing a function $f$ is said to have perfect
  correctness if $f(x) = \Phi_{f}(x)$ for all x.
\end{definition}

\begin{definition}[d-Privacy]
\label{def:mpc:d-privacy}
  An MPC protocol $\Phi_{f}$ is said to have $d$-privacy if $d$ parties colluding
  cannot obtain any information about the private inputs of the remaining $n-d$ parties.

  More formally, the protocol has $d$-privacy if it is possible to define a
  simulator $S_{A}$ which is given access to the output of the protocol,
  producing views that are indistinguishable from the views of the $d$ colluding
  parties.
\end{definition}

%%% Local Variables:
%%% mode: latex
%%% TeX-master: "../main"
%%% End:
