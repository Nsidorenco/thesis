\chapter{Background}
\label{ch:background}


\section{Sigma Protocols}
\label{sec:sigma_protocols}
Originally introduced by Cramer, $\Sigma$-protocols are protocols on a
three-move-form, based on a, computationally hard, relation $R$, such that $(h, w) \in R$
if $h$ is an instance of a computationally hard problem, and $w$ is
the solution to $h$. $\Sigma$-protocols then allows a prover, P, who knows the
solution $w$, to convince a verify, V, of the existence of $w$, without
explicitly showing $w$ to him.

The following section aims to introduce the definition of $\Sigma$-protocols,
along with its notions of security. The following section is based on the
presentation of $\Sigma$-protocols by \citet{on-sigma-protocols}.

To prove security of a $\Sigma$-protocols, we require three properties, namely,
\textbf{Completeness}, \textbf{Special Soundness}, and \textbf{Special Honest Verifier Zero Knowledge (SHVZK)}.

\paragraph{Completeness}
Protocol should always succeed with correct output, if both parties are honest.

\paragraph{Special Soundness}
Given two, accepting, transcripts, with different challenges it is possible to
compute an accepting witness for the statement in the relation.

The special soundness property is important for ensuring that a cheating prover
cannot succeed. Given special soundness, if the protocol is run multiple times,
his advantage becomes negligible, since special soundness implies that there can
only exists one challenge, for any given message $a$, which can make the
protocol accept, without knowing the witness. Therefore, given a challenge space
with cardinality $c$, the probability of a cheating prover succeeding in
convincing the verifier is $\frac{1}{c}$. The protocol can then be run multiple
times, to ensure negligible probability.

Can also be generalised to $s$-Special Soundness, which requires that the
witness can be constructed, given $s$ accepting conversations.

\paragraph{SHVZK}
\begin{itemize}
  \item Exists an polynomial-time simulator M.
  \item Given statement, x, and challenge, e, output an accepting
    conversation (a, e, z).
  \item Conversation should have the same distribution as an conversation
    between honest parties.
\end{itemize}

\section{Commitment Schemes}
\label{sec:commitment}

\subsection{Definition}
\label{sec:commitment:definition}



%%% Local Variables:
%%% mode: latex
%%% TeX-master: "../main"
%%% End:
