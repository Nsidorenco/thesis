\chapter{Background}
\label{ch:background}
This section aims to introduce some of the fundamental definitions and concepts
used throughout this thesis. This section will foremost give a rudimentary and
informal introduction, while the later chapters will provide more rigours
formalisations and proofs of security.
First, we start by describing $\Sigma$-protocols, along with its security
definitions and extensions.

\section{Zero-Knowledge}
\label{sec:background:zk}


\section{Sigma Protocols}
\label{sec:background:sigma_protocols}
Originally introduced by Cramer \todo{reference}, $\Sigma$-protocols are two-party protocols with a
three-move-form, based on a, computationally hard, relation $R$, such that $(h, w) \in R$
if $h$ is an instance of a computationally hard problem, and $w$ is
the solution to $h$. $\Sigma$-protocols then allows a prover, P, who knows the
solution $w$, to convince a verify, V, of the existence of $w$, without
explicitly showing $w$ to him.

\begin{figure}[ht]
  \centering
  \begin{tikzpicture}
      % \draw (-3,0) -- (-3,-3) (3,0) -- (3,-3);
      \node at (-3,.3) {Prover};
      \node at (3,.3) {Verifier};
      \draw[->] (-3,-1) -- node[midway,above] {a} (3,-1);
      \draw[<-] (-3,-2) -- node[midway,above] {e} (3,-2);
      \draw[->] (-3,-3) -- node[midway,above] {z} (3,-3);
  \end{tikzpicture}
  \caption{\label{fig:proto_sigma} $\Sigma$-Protocol}
\end{figure}


The following section aims to introduce the definition of $\Sigma$-protocols,
along with its notions of security. The following section is based on the
presentation of $\Sigma$-protocols by \citet{on-sigma-protocols}.

\todo{explain the flow of the protocol. what is a \dots}

\begin{definition}[$\Sigma$-Protocol Security]
\label{def:sigma:sec}
To prove security of a $\Sigma$-protocols, we require three properties, namely,
\textbf{Completeness}, \textbf{Special Soundness}, and \textbf{Special Honest Verifier Zero Knowledge (SHVZK)}.
\end{definition}

\paragraph{Completeness}
Protocol should always succeed with correct output, if both parties are honest.

\paragraph{Special Soundness}
Given two, accepting, transcripts, with different challenges it is possible to
compute an accepting witness for the statement in the relation.

The special soundness property is important for ensuring that a cheating prover
cannot succeed. Given special soundness, if the protocol is run multiple times,
his advantage becomes negligible, since special soundness implies that there can
only exists one challenge, for any given message $a$, which can make the
protocol accept, without knowing the witness. Therefore, given a challenge space
with cardinality $c$, the probability of a cheating prover succeeding in
convincing the verifier is $\frac{1}{c}$. The protocol can then be run multiple
times, to ensure negligible probability.

Can also be generalised to $s$-Special Soundness, which requires that the
witness can be constructed, given $s$ accepting conversations.

\paragraph{SHVZK}
Zero knowledge is proven by giving a simulator. Not possible to define for
corrupted verifier, since the challenge can become exponentially long.
\begin{itemize}
  \item Exists an polynomial-time simulator M.
  \item Given statement, x, and challenge, e, output an accepting
    conversation (a, e, z).
  \item Conversation should have the same distribution as an conversation
    between honest parties.
\end{itemize}

\todo{Compound protocols}

\todo{Relation in literature not always correct}

\todo{non-interactive by Fiat-Shamir}

\section{Commitment Schemes}
\label{sec:background:commitment}
Commitment schemes is another fundamental building block in cryptography, and
has a strong connection to $\Sigma$-Protocols \todo{reference}. Commitment
schemes facilitates an interaction between two parties, P1 and P2, where P1 can
generate a commitment of a message, which he can send to P2, without revealing
what his original message where. At a later point P1 can then send the message
to P2, who is then able to verify that P1 has not altered his message since
creating the commitment. More formally a commitment schemes is defined as:

\begin{definition}[Commitment Schemes]
A commitment schemes is a tuple of algorithms (KeyG, Com, Ver), where:
\begin{itemize}
  \item $(ck, vk) \leftarrow KeyG()$, provides key generation.
  \item $(c, d) \leftarrow Com(ck, m)$ generates a commitment $c$ of the message
    $m$ along with a opening key $d$, which can be revealed at a later time.
  \item $\{true, false\} \leftarrow Ver(vk, c, m, d)$ checked whether the
    commitment $c$ was generated from $m$ and opening key $d$.
\end{itemize}
\end{definition}

For a commitment schemes to be secure it is required to satisfy three
properties: \textbf{Correctness}, \textbf{Binding}, and \textbf{Hiding}.

\paragraph{Correctness}
A commitment scheme is said to be correct, if a commitment $(c, d)$ made by a
honest party will always be accepted by the verification procedure of another
party, i.e:
$$
Pr[Ver(vk, c, m, d) | (c, d) = Com(ck, m) \land (ck, vk) \leftarrow KeyG()] = 1.
$$

\paragraph{Binding}
The binding property states that a party committing to a message will not be
able to successfully convince another party, that he has committed to another
message, i.e. $(c, d) \leftarrow Com(ck, m)$, will not be able to find an alternative opening key $d'$
and message $m'$ such that $(c, d') \leftarrow Com(ck, m')$.

\todo{write binding game}

\todo{Stat, perfect, comp variants}

\todo{All games parametrised by security param. Not used in formal definition}

\paragraph{Hiding}
The Hiding property states that a party given a commitment $c$, will not be able
to extract the message $m$, which the commitment was based on.

\section{Multi-Part Computation (MPC)}
\label{sec:background:mpc}
Consider the problem, where $n$ parties, called $P_{1}, \dots, p_{n}$, with
corresponding input values $\textbf{x} = x_{1}, \dots, x_{n}$ want to compute a commonly
known function, $f : (\text{input})^{n} \rightarrow \text{output}$, such that
every party agrees on the same output $y$, such that $y = f(\textbf{x})$, but none of them learns the inputs to function, barring their own.

\todo{Different notions of security. We need passive.}

This is the exact problem that MPC aims to solve.

\begin{definition}[Correctness]
\label{def:mpc:correctness}

\end{definition}

\begin{definition}[d-Privacy]
\label{def:mpc:d-privacy}

\end{definition}

%%% Local Variables:
%%% mode: latex
%%% TeX-master: "../main"
%%% End:
