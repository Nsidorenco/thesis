\chapter{Formalising ZKBoo}
\label{ch:formal_zkboo}

To formalise the ZKBoo protocol we need the following:

\begin{itemize}
  \item A formalisation of Arithmetic circuits
    \begin{itemize}
        \item With evaluation semantics
    \end{itemize}
  \item A formalisation of the (2,3)-Decomposition
    \begin{itemize}
      \item Proof of completeness
      \item Proof of 2-Privacy
    \end{itemize}
  \item An instantiation of a $\Sigma$-protocol for ``MPC in the head''
\end{itemize}

\section{Formalising Arithmetic circuits}
\label{sec:arith_circuits}
Since the function we are computing are on the form $\phi(x) = y$, where $\phi$
can perform any arbitrary computation depending on x we need gates with
out-degree of $\geq 1$.

We are restricted to finite integer rings, and therefore, we only need four
basic gates to express our circuit, namely, Add to constant (ADDC),
multiply by constant (MULTC), addition of two wires, and multiplication
of two wires.

For ADDC and MULTC the gates only have one in-going edge, but
can have arbitrary out-going edges. ADD and MULT have two in-going edges, but
can also have an arbitrary number of out-going edges.

\paragraph{Depth of circuits}
Since arithmetic circuits can usually be expressed as an upside-down tree, the
depth of the circuit is depth of the tree.
We chose the visualize our circuit is a different way, namely that, each gate in
the circuit is its own layers and the depth of the circuit is therefore the
number of gates in the circuit. The reason for this, is to simplify the proofs
of correctness and privacy, which both depend on the structure on the circuits.
\todo{This can be seen in the following sections \dots}.

\todo{Tikz example showing the two representations}

From this representation each gate can be labelled according to its depth. This
gives an ordering on the execution for all the gates. From this ordering a
natural choice is the represent the circuit as a list of gates, such that gate
at depth 0 is the gate at index 0 in the list.

One remaining problem is then, that gates can have input wires coming from any
depth of the circuit, and the is no way to calculate the incoming gates based on
its location within the list. We therefore encode the incoming wires directly
into our representation of the gates, e.g. the multiplication gate will consist
of a tuple $(i,j)$ where $i$ is the index of left incoming wire and $j$ is the
index of right incoming wire.

\begin{lstlisting}[float,label=lst:gate_types,caption=Type declaration of gates]
type gate = [
  | ADDC of (int * int)
  | MULTC of (int * int)
  | MULT of (int * int)
  | ADD of (int * int)
].

type circuit = gate list.
\end{lstlisting}

To differentiate between the tuples we use \easycrypt's type system to make
types for each gate, which can be seen in figure \ref{lst:gate_types}. Here the
unary gates of ADDC and MULTC are also represented as tuples. The first entry of
the tuple is the incoming wire and the second entry is the constant used.

\begin{lstlisting}[float,label=lst:circuit_eval,caption=Circuit evaluation function]
op eval_gate (g : gate, s : int list) : int =
  with g = MULT inputs => let (i, j) = inputs in
                          let x = (nth 0 s i) in
                          let y = (nth 0 s j) in x * y
  with g = ADD inputs =>  let (i, j) = inputs in
                          let x = (nth 0 s i) in
                          let y = (nth 0 s j) in x + y
  with g = ADDC inputs => let (i, c) = inputs in
                          let x = (nth 0 s i) in x + c
  with g = MULTC inputs => let (i, c) = inputs in
                          let x = (nth 0 s i) in x * c.

op eval_circuit_aux(c : circuit, s : int list) : int list =
    with c = [] => s
    with c = g :: gs =>
     let r = eval_gate g s in
     eval_circuit_aux gs (rcons s r).

op eval_circuit (c : circuit, s : state) : output =
    last 0 (eval_circuit_aux c s).
\end{lstlisting}

From this representation of circuits as a list of gates, where gates are types,
it is not possible to define the semantic meaning of this representation, by
defining the evaluation function, which can be seen in figure \ref{lst:circuit_eval}.
The evaluation is broken into two parts: one for evaluation one gate, and one
for evaluating the entire circuits, based on the former.
To evaluate one gate, we first need to determine which gate it is. This is done
by matching on the type of the gate.
Then, since execution order follows the ordering of indexes, it can be assumed
that if we are computing index $i$ of the circuit, then indices $[0 \dots i-1]$
have already been computed to an value representing the result of computing the
gate. Perform the appropriate function then simply reduces to looking the values
of the incoming wires and computing the function.

Computing the entire gate then follows from the same fact, that gate are always
evaluated in the order the appear in the list, and the no gate can depend on
the result of gates, which have a higher index than itself. By continually
performing gate evaluation of the first entry of the list, saving the result
into ``state'' where each index correspond to the computed value of the gate at
that index in the circuit, and the calling recursively on the list with the
first entry removed, then the output of the gate will be in the last entry of
the state, when there are no more gates to compute. Assuming that there is only
one output gate.

\todo{Add function to verify correctness of circuit?}

\section{(2,3) Decomposition of circuits}
\label{sec:decomposition}

\todo{We put no assumptions on the circuit being valid, since the decomposition
  will fail with the same values as the circuit evaluation.}



%%% Local Variables:
%%% mode: latex
%%% TeX-master: "../main"
%%% End:
