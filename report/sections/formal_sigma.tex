\chapter{Formalising Sigma-Protocols}
\label{ch:formal_sigma}
This section will aim to formalise $\Sigma$-protocols according to the
definitions set out in section \ref{sec:background:sigma_protocols}, with a
sufficiently general set-up to allows easy instantiation of arbitrary concrete
protocols.
This abstraction is provided by \easycrypt's abstract theories, where we let
$\Sigma$-Protocol be quantified over the abstract types:
\begin{itemize}
  \item message
  \item witness
  \item response
  \item randomness
  \item input
\end{itemize}

And a relation $R : (\text{input} \times \text{witness}) \rightarrow \{0,1\}$.

We then define the $\Sigma$-protocol itself be a series of probabilistic procedures:
\begin{itemize}
  \item init
  \item response
  \item verify
\end{itemize}

\todo{Rewrite as EC module}

Along with another set of algorithms required to satisfy the definition of
security defined in \ref{def:sigma:sec}.

\begin{itemize}
  \item gen
  \item witness\_extractor
  \item simulator
\end{itemize}

An instantiation of a $\Sigma$-Protocol is the an implementation of the above procedures.

We then model security as a series of games:

\begin{definition}[Completeness]
  \begin{lstlisting}[float, label=lst:sigma_completeness,caption=Completeness game for $\Sigma$-Protocols]
  module Completeness(S : SigmaProtocol) = {
    proc main(h : input, w : witness) : bool = {
       var a, e, z;
       a = S.init(h,w);
       e <$ dchallenge;
       z = S.response(h, a, e);
       v = S.verify(h, a, e, z);
       return v;
    }
  }.
  \end{lstlisting}
  We say the protocol is complete, if the probabilistic procedure in
  \ref{lst:sigma_completeness} outputs 1 with probability 1, i.e.
  \begin{equation}
    \label{eq:sigma_completeness}
    \forall \&m h w, R h w \implies Pr[\text{Completeness.main}(h,w) @ \&m : res] = 1\%r.
  \end{equation}
\end{definition}

\begin{definition}[Special Soundness]

\end{definition}

\begin{lemma}
  A $\Sigma$-Protocol is secure if all the above games succeed with probability $1$
\end{lemma}

\todo{Argue that games corresponds to original definitions}

\todo{To prove composition we assume to following relations to be true \dots and
this only hold if both inputs are in the domain of R.}

%%% Local Variables:
%%% mode: latex
%%% TeX-master: "../main"
%%% End:
