% Created 2020-01-29 Wed 12:33
% Intended LaTeX compiler: pdflatex
\documentclass[11pt]{article}
\usepackage[utf8]{inputenc}
\usepackage[T1]{fontenc}
\usepackage{graphicx}
\usepackage{grffile}
\usepackage{longtable}
\usepackage{wrapfig}
\usepackage{rotating}
\usepackage[normalem]{ulem}
\usepackage{amsmath}
\usepackage{textcomp}
\usepackage{amssymb}
\usepackage{capt-of}
\usepackage{hyperref}
\bibliographystyle{plain}
\bibliography{report/refs}
\author{Nikolaj Sidorenco}
\date{\today}
\title{Project Description}
\hypersetup{
 pdfauthor={Nikolaj Sidorenco},
 pdftitle={Project Description},
 pdfkeywords={},
 pdfsubject={},
 pdfcreator={Emacs 26.3 (Org mode 9.4)}, 
 pdflang={English}}
\begin{document}

\maketitle

\section{Project}
\label{sec:orgd37f878}

In this project I aim to formalise Sigma-protocols and commitment schemes within
EasyCrypt with the help of existing formalisations.
Moreover, I seek to understand the differeces between the available
cryptographic proof assistents by comparing my formalisations to existings ones
in the cryptHOL proof assistent.


When fomalising Sigma-protocols I also aim to prove general constructions, like
the OR-construction, and proving them the be secure. Having formalised both
Sigma-Protocols and Commitment schemes I will provide several concrete instantiations
and prove their security in presence of passively corrupted parties.


Time permitting I will either prove the relation between Sigma protocols and
Commitment schemes, the Fiat-Shamir protocol for non-interactive Sigma-protocols
or other possible extensions.


\section{Literature}
\label{sec:orgbd9b260}
\begin{itemize}
\item Butler, D., Lochbihler, A., Aspinall, D., \& Gascon, A. (2019). Formalising
\(\Sigma\)-protocols and commitment schemes using cryptHOL.
\item Metere, R., \& Dong, C. (2017). Automated cryptographic analysis of the
Pedersen commitment scheme. CoRR, abs/1705.05897(), .
\item Damgård, I. (2011). On \(\Sigma\)-protocols.
\end{itemize}
\end{document}
